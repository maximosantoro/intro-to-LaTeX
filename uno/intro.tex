\documentclass{article}
\usepackage[utf8]{inputenc}
\usepackage{amsmath}


\title{Intro to \LaTeX}
\author{Maximo Santoro}
\date{October 2020}

\begin{document}

\maketitle

\section{Introduction}
Hoy voy a aprender a usar \LaTeX.

\section{Euler}
\begin{itemize}

\item Formula de Euler: 

$e^{i\pi}+1 = 0$

\item Tenemos 

$$ e= \lim_{n\to\infty} \left(1+\frac{1}{n}\right)^n  =  
\lim_{n_\to\infty}\frac{n}{\sqrt[n]{n!}}$$

\item Ahora:

$$e=\sum_{n=0}^{\infty} \frac{1}{n!}$$

\item Tambien:
$$ e=2+\frac{1}{1+\frac{1}{2+\frac{2}{3+\frac{3}{4+\frac{4}{5+\ddots}}}}} $$
\end{itemize}

\section{Generales}

\begin{itemize}

\item Integral simple:
$$\int_a^bf(x)dx$$

\item Integral triple:
$$\iiint f(x,y,z)dxdydz$$

\item Vector:
$$\vec{v}={v_1, v_2, v_3}$$

\item Producto interno:
$$\vec{v}\cdot \vec{w}$$

\item Una matriz:
$$\begin{bmatrix}
1 & 2 & 3\\
4 & 5 & 6\\
7 & 8 & 9\\
\end{bmatrix}
$$

\end{itemize}
\end{document}
